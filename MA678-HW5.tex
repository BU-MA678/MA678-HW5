% Options for packages loaded elsewhere
\PassOptionsToPackage{unicode}{hyperref}
\PassOptionsToPackage{hyphens}{url}
%
\documentclass[
]{article}
\usepackage{amsmath,amssymb}
\usepackage{iftex}
\ifPDFTeX
  \usepackage[T1]{fontenc}
  \usepackage[utf8]{inputenc}
  \usepackage{textcomp} % provide euro and other symbols
\else % if luatex or xetex
  \usepackage{unicode-math} % this also loads fontspec
  \defaultfontfeatures{Scale=MatchLowercase}
  \defaultfontfeatures[\rmfamily]{Ligatures=TeX,Scale=1}
\fi
\usepackage{lmodern}
\ifPDFTeX\else
  % xetex/luatex font selection
\fi
% Use upquote if available, for straight quotes in verbatim environments
\IfFileExists{upquote.sty}{\usepackage{upquote}}{}
\IfFileExists{microtype.sty}{% use microtype if available
  \usepackage[]{microtype}
  \UseMicrotypeSet[protrusion]{basicmath} % disable protrusion for tt fonts
}{}
\makeatletter
\@ifundefined{KOMAClassName}{% if non-KOMA class
  \IfFileExists{parskip.sty}{%
    \usepackage{parskip}
  }{% else
    \setlength{\parindent}{0pt}
    \setlength{\parskip}{6pt plus 2pt minus 1pt}}
}{% if KOMA class
  \KOMAoptions{parskip=half}}
\makeatother
\usepackage{xcolor}
\usepackage[margin=1in]{geometry}
\usepackage{color}
\usepackage{fancyvrb}
\newcommand{\VerbBar}{|}
\newcommand{\VERB}{\Verb[commandchars=\\\{\}]}
\DefineVerbatimEnvironment{Highlighting}{Verbatim}{commandchars=\\\{\}}
% Add ',fontsize=\small' for more characters per line
\usepackage{framed}
\definecolor{shadecolor}{RGB}{248,248,248}
\newenvironment{Shaded}{\begin{snugshade}}{\end{snugshade}}
\newcommand{\AlertTok}[1]{\textcolor[rgb]{0.94,0.16,0.16}{#1}}
\newcommand{\AnnotationTok}[1]{\textcolor[rgb]{0.56,0.35,0.01}{\textbf{\textit{#1}}}}
\newcommand{\AttributeTok}[1]{\textcolor[rgb]{0.13,0.29,0.53}{#1}}
\newcommand{\BaseNTok}[1]{\textcolor[rgb]{0.00,0.00,0.81}{#1}}
\newcommand{\BuiltInTok}[1]{#1}
\newcommand{\CharTok}[1]{\textcolor[rgb]{0.31,0.60,0.02}{#1}}
\newcommand{\CommentTok}[1]{\textcolor[rgb]{0.56,0.35,0.01}{\textit{#1}}}
\newcommand{\CommentVarTok}[1]{\textcolor[rgb]{0.56,0.35,0.01}{\textbf{\textit{#1}}}}
\newcommand{\ConstantTok}[1]{\textcolor[rgb]{0.56,0.35,0.01}{#1}}
\newcommand{\ControlFlowTok}[1]{\textcolor[rgb]{0.13,0.29,0.53}{\textbf{#1}}}
\newcommand{\DataTypeTok}[1]{\textcolor[rgb]{0.13,0.29,0.53}{#1}}
\newcommand{\DecValTok}[1]{\textcolor[rgb]{0.00,0.00,0.81}{#1}}
\newcommand{\DocumentationTok}[1]{\textcolor[rgb]{0.56,0.35,0.01}{\textbf{\textit{#1}}}}
\newcommand{\ErrorTok}[1]{\textcolor[rgb]{0.64,0.00,0.00}{\textbf{#1}}}
\newcommand{\ExtensionTok}[1]{#1}
\newcommand{\FloatTok}[1]{\textcolor[rgb]{0.00,0.00,0.81}{#1}}
\newcommand{\FunctionTok}[1]{\textcolor[rgb]{0.13,0.29,0.53}{\textbf{#1}}}
\newcommand{\ImportTok}[1]{#1}
\newcommand{\InformationTok}[1]{\textcolor[rgb]{0.56,0.35,0.01}{\textbf{\textit{#1}}}}
\newcommand{\KeywordTok}[1]{\textcolor[rgb]{0.13,0.29,0.53}{\textbf{#1}}}
\newcommand{\NormalTok}[1]{#1}
\newcommand{\OperatorTok}[1]{\textcolor[rgb]{0.81,0.36,0.00}{\textbf{#1}}}
\newcommand{\OtherTok}[1]{\textcolor[rgb]{0.56,0.35,0.01}{#1}}
\newcommand{\PreprocessorTok}[1]{\textcolor[rgb]{0.56,0.35,0.01}{\textit{#1}}}
\newcommand{\RegionMarkerTok}[1]{#1}
\newcommand{\SpecialCharTok}[1]{\textcolor[rgb]{0.81,0.36,0.00}{\textbf{#1}}}
\newcommand{\SpecialStringTok}[1]{\textcolor[rgb]{0.31,0.60,0.02}{#1}}
\newcommand{\StringTok}[1]{\textcolor[rgb]{0.31,0.60,0.02}{#1}}
\newcommand{\VariableTok}[1]{\textcolor[rgb]{0.00,0.00,0.00}{#1}}
\newcommand{\VerbatimStringTok}[1]{\textcolor[rgb]{0.31,0.60,0.02}{#1}}
\newcommand{\WarningTok}[1]{\textcolor[rgb]{0.56,0.35,0.01}{\textbf{\textit{#1}}}}
\usepackage{graphicx}
\makeatletter
\def\maxwidth{\ifdim\Gin@nat@width>\linewidth\linewidth\else\Gin@nat@width\fi}
\def\maxheight{\ifdim\Gin@nat@height>\textheight\textheight\else\Gin@nat@height\fi}
\makeatother
% Scale images if necessary, so that they will not overflow the page
% margins by default, and it is still possible to overwrite the defaults
% using explicit options in \includegraphics[width, height, ...]{}
\setkeys{Gin}{width=\maxwidth,height=\maxheight,keepaspectratio}
% Set default figure placement to htbp
\makeatletter
\def\fps@figure{htbp}
\makeatother
\setlength{\emergencystretch}{3em} % prevent overfull lines
\providecommand{\tightlist}{%
  \setlength{\itemsep}{0pt}\setlength{\parskip}{0pt}}
\setcounter{secnumdepth}{-\maxdimen} % remove section numbering
\ifLuaTeX
  \usepackage{selnolig}  % disable illegal ligatures
\fi
\IfFileExists{bookmark.sty}{\usepackage{bookmark}}{\usepackage{hyperref}}
\IfFileExists{xurl.sty}{\usepackage{xurl}}{} % add URL line breaks if available
\urlstyle{same}
\hypersetup{
  pdftitle={MA678 Homework 5},
  pdfauthor={Yuchen Huang},
  hidelinks,
  pdfcreator={LaTeX via pandoc}}

\title{MA678 Homework 5}
\author{Yuchen Huang}
\date{2023-10-20}

\begin{document}
\maketitle

\hypertarget{poisson-and-negative-binomial-regression}{%
\subsection{15.1 Poisson and negative binomial
regression}\label{poisson-and-negative-binomial-regression}}

The folder \texttt{RiskyBehavior} contains data from a randomized trial
targeting couples at high risk of HIV infection. The intervention
provided counseling sessions regarding practices that could reduce their
likelihood of contracting HIV. Couples were randomized either to a
control group, a group in which just the woman participated, or a group
in which both members of the couple participated. One of the outcomes
examined after three months was ``number of unprotected sex acts.''

\hypertarget{a}{%
\subsubsection{a)}\label{a}}

Model this outcome as a function of treatment assignment using a Poisson
regression. Does the model fit well? Is there evidence of
overdispersion?

\begin{Shaded}
\begin{Highlighting}[]
\NormalTok{risky }\OtherTok{\textless{}{-}} \FunctionTok{read.csv}\NormalTok{(}\StringTok{"risky.csv"}\NormalTok{, }\AttributeTok{header =}\NormalTok{ T)}
\CommentTok{\#create a new row called treatment. 3  = couple, 2 = women alone, 1 = education}
\NormalTok{risky }\OtherTok{\textless{}{-}}\NormalTok{ risky }\SpecialCharTok{|\textgreater{}}
    \FunctionTok{mutate}\NormalTok{(}\AttributeTok{treatment =} \FunctionTok{ifelse}\NormalTok{(couples }\SpecialCharTok{==} \DecValTok{1}\NormalTok{, }\DecValTok{3}\NormalTok{, }\FunctionTok{ifelse}\NormalTok{(women\_alone }\SpecialCharTok{==} \DecValTok{1}\NormalTok{, }\DecValTok{2}\NormalTok{, }\DecValTok{1}\NormalTok{))) }\SpecialCharTok{|\textgreater{}}
    \FunctionTok{mutate}\NormalTok{(}\AttributeTok{fupacts =} \FunctionTok{round}\NormalTok{(fupacts))}
\NormalTok{m }\OtherTok{\textless{}{-}}  \FunctionTok{stan\_glm}\NormalTok{(}\AttributeTok{data =}\NormalTok{ risky, fupacts}\SpecialCharTok{\textasciitilde{}}\NormalTok{treatment, }\AttributeTok{family =} \FunctionTok{poisson}\NormalTok{(}\AttributeTok{link=}\StringTok{"log"}\NormalTok{), }\AttributeTok{refresh =} \DecValTok{0}\NormalTok{)}
\FunctionTok{summary}\NormalTok{(m)}
\end{Highlighting}
\end{Shaded}

\begin{verbatim}
## 
## Model Info:
##  function:     stan_glm
##  family:       poisson [log]
##  formula:      fupacts ~ treatment
##  algorithm:    sampling
##  sample:       4000 (posterior sample size)
##  priors:       see help('prior_summary')
##  observations: 434
##  predictors:   2
## 
## Estimates:
##               mean   sd   10%   50%   90%
## (Intercept)  3.1    0.0  3.1   3.1   3.2 
## treatment   -0.2    0.0 -0.2  -0.2  -0.1 
## 
## Fit Diagnostics:
##            mean   sd   10%   50%   90%
## mean_PPD 16.5    0.3 16.1  16.5  16.8 
## 
## The mean_ppd is the sample average posterior predictive distribution of the outcome variable (for details see help('summary.stanreg')).
## 
## MCMC diagnostics
##               mcse Rhat n_eff
## (Intercept)   0.0  1.0  2817 
## treatment     0.0  1.0  2748 
## mean_PPD      0.0  1.0  2959 
## log-posterior 0.0  1.0  1843 
## 
## For each parameter, mcse is Monte Carlo standard error, n_eff is a crude measure of effective sample size, and Rhat is the potential scale reduction factor on split chains (at convergence Rhat=1).
\end{verbatim}

\begin{Shaded}
\begin{Highlighting}[]
\FunctionTok{pp\_check}\NormalTok{(m)}
\end{Highlighting}
\end{Shaded}

\includegraphics{MA678-HW5_files/figure-latex/unnamed-chunk-1-1.pdf}

From the posterior predictive check plot, we can see our model predicts
less 0s than the real data,so there might be 0 inflation.

\begin{Shaded}
\begin{Highlighting}[]
\CommentTok{\#Use reisidual plot to see the dispersion}
\FunctionTok{plot}\NormalTok{(}\FunctionTok{fitted}\NormalTok{(m), }\FunctionTok{resid}\NormalTok{(m), }\AttributeTok{pch =} \DecValTok{20}\NormalTok{, }\AttributeTok{main =} \StringTok{"Residual Plot"}\NormalTok{)}
\end{Highlighting}
\end{Shaded}

\includegraphics{MA678-HW5_files/figure-latex/unnamed-chunk-2-1.pdf}

From the residual plot, we can see a huge amount of data points are
above 0, which indicates that the it is overdispersion. Also, through
the dispersion test, the result is 44 which is a large number indicating
it's overdispersion.

\hypertarget{b}{%
\subsubsection{b)}\label{b}}

Next extend the model to include pre-treatment measures of the outcome
and the additional pre-treatment variables included in the dataset. Does
the model fit well? Is there evidence of overdispersion?

\begin{Shaded}
\begin{Highlighting}[]
\CommentTok{\#indicators are treatment and bs\_hiv}
\NormalTok{m1 }\OtherTok{\textless{}{-}} \FunctionTok{stan\_glm}\NormalTok{(}\AttributeTok{data =}\NormalTok{ risky, fupacts}\SpecialCharTok{\textasciitilde{}}\NormalTok{treatment}\SpecialCharTok{+}\NormalTok{bs\_hiv, }\AttributeTok{family =} \FunctionTok{poisson}\NormalTok{(}\AttributeTok{link=}\StringTok{"log"}\NormalTok{), }\AttributeTok{refresh =} \DecValTok{0}\NormalTok{)}
\FunctionTok{summary}\NormalTok{(m1)}
\end{Highlighting}
\end{Shaded}

\begin{verbatim}
## 
## Model Info:
##  function:     stan_glm
##  family:       poisson [log]
##  formula:      fupacts ~ treatment + bs_hiv
##  algorithm:    sampling
##  sample:       4000 (posterior sample size)
##  priors:       see help('prior_summary')
##  observations: 434
##  predictors:   3
## 
## Estimates:
##                  mean   sd   10%   50%   90%
## (Intercept)     3.2    0.0  3.1   3.2   3.2 
## treatment      -0.1    0.0 -0.2  -0.1  -0.1 
## bs_hivpositive -0.6    0.0 -0.6  -0.6  -0.5 
## 
## Fit Diagnostics:
##            mean   sd   10%   50%   90%
## mean_PPD 16.5    0.3 16.1  16.5  16.9 
## 
## The mean_ppd is the sample average posterior predictive distribution of the outcome variable (for details see help('summary.stanreg')).
## 
## MCMC diagnostics
##                mcse Rhat n_eff
## (Intercept)    0.0  1.0  3790 
## treatment      0.0  1.0  3719 
## bs_hivpositive 0.0  1.0  3075 
## mean_PPD       0.0  1.0  3248 
## log-posterior  0.0  1.0  1811 
## 
## For each parameter, mcse is Monte Carlo standard error, n_eff is a crude measure of effective sample size, and Rhat is the potential scale reduction factor on split chains (at convergence Rhat=1).
\end{verbatim}

\begin{Shaded}
\begin{Highlighting}[]
\FunctionTok{pp\_check}\NormalTok{(m1)}
\end{Highlighting}
\end{Shaded}

\includegraphics{MA678-HW5_files/figure-latex/unnamed-chunk-3-1.pdf}

We can still observe our model create less 0s than the actual data.

\begin{Shaded}
\begin{Highlighting}[]
\FunctionTok{plot}\NormalTok{(}\FunctionTok{fitted}\NormalTok{(m1), }\FunctionTok{resid}\NormalTok{(m1), }\AttributeTok{pch =} \DecValTok{20}\NormalTok{, }\AttributeTok{main =} \StringTok{"Residual Plot"}\NormalTok{)}
\end{Highlighting}
\end{Shaded}

\includegraphics{MA678-HW5_files/figure-latex/unnamed-chunk-4-1.pdf}

From the dispersion test and residual plot, we can see it's still
overdispersion since there are many points far away from 0.

\hypertarget{c}{%
\subsubsection{c)}\label{c}}

Fit a negative binomial (overdispersed Poisson) model. What do you
conclude regarding effectiveness of the intervention?

\begin{Shaded}
\begin{Highlighting}[]
\NormalTok{nb }\OtherTok{\textless{}{-}} \FunctionTok{stan\_glm}\NormalTok{(fupacts}\SpecialCharTok{\textasciitilde{}}\NormalTok{treatment}\SpecialCharTok{+}\NormalTok{bs\_hiv, }\AttributeTok{data =}\NormalTok{ risky, }
               \AttributeTok{family =} \FunctionTok{neg\_binomial\_2}\NormalTok{(}\AttributeTok{link =} \StringTok{"log"}\NormalTok{), }\AttributeTok{refresh =} \DecValTok{0}\NormalTok{)}
\FunctionTok{summary}\NormalTok{(nb)}
\end{Highlighting}
\end{Shaded}

\begin{verbatim}
## 
## Model Info:
##  function:     stan_glm
##  family:       neg_binomial_2 [log]
##  formula:      fupacts ~ treatment + bs_hiv
##  algorithm:    sampling
##  sample:       4000 (posterior sample size)
##  priors:       see help('prior_summary')
##  observations: 434
##  predictors:   3
## 
## Estimates:
##                         mean   sd   10%   50%   90%
## (Intercept)            3.1    0.2  2.8   3.1   3.4 
## treatment             -0.1    0.1 -0.2  -0.1   0.0 
## bs_hivpositive        -0.6    0.2 -0.8  -0.6  -0.3 
## reciprocal_dispersion  0.3    0.0  0.3   0.3   0.4 
## 
## Fit Diagnostics:
##            mean   sd   10%   50%   90%
## mean_PPD 16.7    2.0 14.2  16.5  19.4 
## 
## The mean_ppd is the sample average posterior predictive distribution of the outcome variable (for details see help('summary.stanreg')).
## 
## MCMC diagnostics
##                       mcse Rhat n_eff
## (Intercept)           0.0  1.0  4588 
## treatment             0.0  1.0  4568 
## bs_hivpositive        0.0  1.0  4604 
## reciprocal_dispersion 0.0  1.0  4453 
## mean_PPD              0.0  1.0  4265 
## log-posterior         0.0  1.0  1955 
## 
## For each parameter, mcse is Monte Carlo standard error, n_eff is a crude measure of effective sample size, and Rhat is the potential scale reduction factor on split chains (at convergence Rhat=1).
\end{verbatim}

\begin{Shaded}
\begin{Highlighting}[]
\FunctionTok{pp\_check}\NormalTok{(nb)}
\end{Highlighting}
\end{Shaded}

\includegraphics{MA678-HW5_files/figure-latex/unnamed-chunk-5-1.pdf}

\begin{Shaded}
\begin{Highlighting}[]
\FunctionTok{exp}\NormalTok{(}\FunctionTok{coef}\NormalTok{(nb)[}\DecValTok{2}\NormalTok{])}
\end{Highlighting}
\end{Shaded}

\begin{verbatim}
## treatment 
## 0.9197329
\end{verbatim}

The model fits better, but the model contains more 0s than the real
data.

The coefficient shows that if the treatment applied, unprotected sex
will decrease by 8.2\%.

\hypertarget{d}{%
\subsubsection{d)}\label{d}}

These data include responses from both men and women from the
participating couples. Does this give you any concern with regard to our
modeling assumptions?

I think the mode will fit better if the data specify the gender of the
one who received education, with adding one more indicator we might be
able to decrease the 0s in the predicting model.

\hypertarget{binomial-regression}{%
\subsection{15.3 Binomial regression}\label{binomial-regression}}

Redo the basketball shooting example on page 270, making some changes:

\hypertarget{a-1}{%
\subsubsection{(a)}\label{a-1}}

Instead of having each player shoot 20 times, let the number of shots
per player vary, drawn from the uniform distribution between 10 and 30.

\begin{Shaded}
\begin{Highlighting}[]
\CommentTok{\#}
\NormalTok{height }\OtherTok{\textless{}{-}} \FunctionTok{rnorm}\NormalTok{(}\DecValTok{100}\NormalTok{, }\DecValTok{72}\NormalTok{, }\DecValTok{3}\NormalTok{)}
\NormalTok{p }\OtherTok{\textless{}{-}} \FloatTok{0.4} \SpecialCharTok{+} \FloatTok{0.1}\SpecialCharTok{*}\NormalTok{(height}\DecValTok{{-}72}\NormalTok{)}\SpecialCharTok{/}\DecValTok{3}
\NormalTok{n }\OtherTok{\textless{}{-}} \FunctionTok{round}\NormalTok{(}\FunctionTok{runif}\NormalTok{(}\DecValTok{100}\NormalTok{, }\DecValTok{10}\NormalTok{, }\DecValTok{30}\NormalTok{))}
\NormalTok{y }\OtherTok{\textless{}{-}} \FunctionTok{rbinom}\NormalTok{(}\DecValTok{100}\NormalTok{, n, p )}
\NormalTok{bb }\OtherTok{\textless{}{-}} \FunctionTok{data.frame}\NormalTok{(}\AttributeTok{n =}\NormalTok{ n, }\AttributeTok{y =}\NormalTok{ y, }\AttributeTok{height =}\NormalTok{ height)}
\NormalTok{m }\OtherTok{\textless{}{-}} \FunctionTok{stan\_glm}\NormalTok{(}\FunctionTok{cbind}\NormalTok{(y,n}\SpecialCharTok{{-}}\NormalTok{y) }\SpecialCharTok{\textasciitilde{}}\NormalTok{ height, }\AttributeTok{family =} \FunctionTok{binomial}\NormalTok{(}\AttributeTok{link=}\StringTok{"logit"}\NormalTok{), }\AttributeTok{data =}\NormalTok{ bb,}\AttributeTok{refresh =} \DecValTok{0}\NormalTok{)}
\FunctionTok{summary}\NormalTok{(m)}
\end{Highlighting}
\end{Shaded}

\begin{verbatim}
## 
## Model Info:
##  function:     stan_glm
##  family:       binomial [logit]
##  formula:      cbind(y, n - y) ~ height
##  algorithm:    sampling
##  sample:       4000 (posterior sample size)
##  priors:       see help('prior_summary')
##  observations: 100
##  predictors:   2
## 
## Estimates:
##               mean   sd    10%   50%   90%
## (Intercept) -12.6    1.2 -14.2 -12.5 -11.0
## height        0.2    0.0   0.1   0.2   0.2
## 
## Fit Diagnostics:
##            mean   sd   10%   50%   90%
## mean_PPD 7.8    0.3  7.4   7.8   8.1  
## 
## The mean_ppd is the sample average posterior predictive distribution of the outcome variable (for details see help('summary.stanreg')).
## 
## MCMC diagnostics
##               mcse Rhat n_eff
## (Intercept)   0.0  1.0  2531 
## height        0.0  1.0  2542 
## mean_PPD      0.0  1.0  3062 
## log-posterior 0.0  1.0  1569 
## 
## For each parameter, mcse is Monte Carlo standard error, n_eff is a crude measure of effective sample size, and Rhat is the potential scale reduction factor on split chains (at convergence Rhat=1).
\end{verbatim}

\begin{Shaded}
\begin{Highlighting}[]
\FunctionTok{pp\_check}\NormalTok{(m)}
\end{Highlighting}
\end{Shaded}

\includegraphics{MA678-HW5_files/figure-latex/unnamed-chunk-6-1.pdf}

\hypertarget{b-1}{%
\subsubsection{(b)}\label{b-1}}

Instead of having the true probability of success be linear, have the
true probability be a logistic function, set so that Pr(success) = 0.3
for a player who is 5'9'' and 0.4 for a 6' tall player.

\begin{Shaded}
\begin{Highlighting}[]
\NormalTok{p }\OtherTok{\textless{}{-}} \FunctionTok{invlogit}\NormalTok{(rstanarm}\SpecialCharTok{::}\FunctionTok{logit}\NormalTok{(}\FloatTok{0.4}\NormalTok{) }\SpecialCharTok{+}\NormalTok{ (rstanarm}\SpecialCharTok{::}\FunctionTok{logit}\NormalTok{(}\FloatTok{0.4}\NormalTok{) }\SpecialCharTok{{-}}\NormalTok{ rstanarm}\SpecialCharTok{::}\FunctionTok{logit}\NormalTok{(}\FloatTok{0.3}\NormalTok{))}\SpecialCharTok{/}\DecValTok{3} \SpecialCharTok{*}\NormalTok{(height}\DecValTok{{-}72}\NormalTok{))}
\NormalTok{n }\OtherTok{\textless{}{-}} \FunctionTok{round}\NormalTok{(}\FunctionTok{runif}\NormalTok{(}\DecValTok{100}\NormalTok{,}\DecValTok{10}\NormalTok{,}\DecValTok{30}\NormalTok{), }\DecValTok{0}\NormalTok{)}
\NormalTok{y }\OtherTok{\textless{}{-}} \FunctionTok{rbinom}\NormalTok{(}\DecValTok{100}\NormalTok{, n, p)}
\NormalTok{new\_bb }\OtherTok{\textless{}{-}} \FunctionTok{data.frame}\NormalTok{(n,y,height)}
\NormalTok{m1 }\OtherTok{\textless{}{-}} \FunctionTok{stan\_glm}\NormalTok{(}\FunctionTok{cbind}\NormalTok{(y,n}\SpecialCharTok{{-}}\NormalTok{y) }\SpecialCharTok{\textasciitilde{}}\NormalTok{ height, }\AttributeTok{family =} \FunctionTok{binomial}\NormalTok{(}\AttributeTok{link=}\StringTok{"logit"}\NormalTok{), }\AttributeTok{data =}\NormalTok{ new\_bb, }\AttributeTok{refresh =} \DecValTok{0}\NormalTok{) }
\NormalTok{m1}
\end{Highlighting}
\end{Shaded}

\begin{verbatim}
## stan_glm
##  family:       binomial [logit]
##  formula:      cbind(y, n - y) ~ height
##  observations: 100
##  predictors:   2
## ------
##             Median MAD_SD
## (Intercept) -12.7    1.2 
## height        0.2    0.0 
## 
## ------
## * For help interpreting the printed output see ?print.stanreg
## * For info on the priors used see ?prior_summary.stanreg
\end{verbatim}

\begin{Shaded}
\begin{Highlighting}[]
\FunctionTok{pp\_check}\NormalTok{(m1)}
\end{Highlighting}
\end{Shaded}

\includegraphics{MA678-HW5_files/figure-latex/unnamed-chunk-7-1.pdf}

\hypertarget{tobit-model-for-mixed-discretecontinuous-data}{%
\subsection{15.7 Tobit model for mixed discrete/continuous
data}\label{tobit-model-for-mixed-discretecontinuous-data}}

Experimental data from the National Supported Work example are in the
folder \texttt{Lalonde}. Use the treatment indicator and pre-treatment
variables to predict post-treatment (1978) earnings using a Tobit model.
Interpret the model coefficients.

\begin{Shaded}
\begin{Highlighting}[]
\NormalTok{lalonde }\OtherTok{\textless{}{-}} \FunctionTok{read.dta}\NormalTok{(}\StringTok{"NSW\_dw\_obs.dta"}\NormalTok{)}
\NormalTok{m }\OtherTok{\textless{}{-}} \FunctionTok{vglm}\NormalTok{(}\FunctionTok{log}\NormalTok{(re78}\SpecialCharTok{+}\DecValTok{1}\NormalTok{) }\SpecialCharTok{\textasciitilde{}}\NormalTok{ treat }\SpecialCharTok{+}\NormalTok{ re75, }\FunctionTok{tobit}\NormalTok{(}\AttributeTok{Lower =} \DecValTok{0}\NormalTok{, }\AttributeTok{Upper=}\DecValTok{10}\NormalTok{), }\AttributeTok{data =}\NormalTok{ lalonde, }\AttributeTok{refresh =} \DecValTok{0}\NormalTok{)}
\end{Highlighting}
\end{Shaded}

\begin{verbatim}
## Warning in eval(slot(family, "initialize")): replacing response values >
## 'Upper' by 'Upper'
\end{verbatim}

\begin{verbatim}
## Warning in checkwz(wz, M = M, trace = trace, wzepsilon = control$wzepsilon): 23
## diagonal elements of the working weights variable 'wz' have been replaced by
## 1.819e-12
\end{verbatim}

\begin{verbatim}
## Warning in checkwz(wz, M = M, trace = trace, wzepsilon = control$wzepsilon): 2
## diagonal elements of the working weights variable 'wz' have been replaced by
## 1.819e-12
\end{verbatim}

\begin{verbatim}
## Warning in checkwz(wz, M = M, trace = trace, wzepsilon = control$wzepsilon): 13
## diagonal elements of the working weights variable 'wz' have been replaced by
## 1.819e-12
\end{verbatim}

\begin{verbatim}
## Warning in checkwz(wz, M = M, trace = trace, wzepsilon = control$wzepsilon): 1
## diagonal elements of the working weights variable 'wz' have been replaced by
## 1.819e-12
\end{verbatim}

\begin{verbatim}
## Warning in checkwz(wz, M = M, trace = trace, wzepsilon = control$wzepsilon): 5
## diagonal elements of the working weights variable 'wz' have been replaced by
## 1.819e-12
\end{verbatim}

\begin{verbatim}
## Warning in checkwz(wz, M = M, trace = trace, wzepsilon = control$wzepsilon): 2
## diagonal elements of the working weights variable 'wz' have been replaced by
## 1.819e-12

## Warning in checkwz(wz, M = M, trace = trace, wzepsilon = control$wzepsilon): 2
## diagonal elements of the working weights variable 'wz' have been replaced by
## 1.819e-12

## Warning in checkwz(wz, M = M, trace = trace, wzepsilon = control$wzepsilon): 2
## diagonal elements of the working weights variable 'wz' have been replaced by
## 1.819e-12

## Warning in checkwz(wz, M = M, trace = trace, wzepsilon = control$wzepsilon): 2
## diagonal elements of the working weights variable 'wz' have been replaced by
## 1.819e-12

## Warning in checkwz(wz, M = M, trace = trace, wzepsilon = control$wzepsilon): 2
## diagonal elements of the working weights variable 'wz' have been replaced by
## 1.819e-12

## Warning in checkwz(wz, M = M, trace = trace, wzepsilon = control$wzepsilon): 2
## diagonal elements of the working weights variable 'wz' have been replaced by
## 1.819e-12

## Warning in checkwz(wz, M = M, trace = trace, wzepsilon = control$wzepsilon): 2
## diagonal elements of the working weights variable 'wz' have been replaced by
## 1.819e-12

## Warning in checkwz(wz, M = M, trace = trace, wzepsilon = control$wzepsilon): 2
## diagonal elements of the working weights variable 'wz' have been replaced by
## 1.819e-12

## Warning in checkwz(wz, M = M, trace = trace, wzepsilon = control$wzepsilon): 2
## diagonal elements of the working weights variable 'wz' have been replaced by
## 1.819e-12

## Warning in checkwz(wz, M = M, trace = trace, wzepsilon = control$wzepsilon): 2
## diagonal elements of the working weights variable 'wz' have been replaced by
## 1.819e-12

## Warning in checkwz(wz, M = M, trace = trace, wzepsilon = control$wzepsilon): 2
## diagonal elements of the working weights variable 'wz' have been replaced by
## 1.819e-12

## Warning in checkwz(wz, M = M, trace = trace, wzepsilon = control$wzepsilon): 2
## diagonal elements of the working weights variable 'wz' have been replaced by
## 1.819e-12

## Warning in checkwz(wz, M = M, trace = trace, wzepsilon = control$wzepsilon): 2
## diagonal elements of the working weights variable 'wz' have been replaced by
## 1.819e-12
\end{verbatim}

\begin{Shaded}
\begin{Highlighting}[]
\FunctionTok{summary}\NormalTok{(m)}
\end{Highlighting}
\end{Shaded}

\begin{verbatim}
## 
## Call:
## vglm(formula = log(re78 + 1) ~ treat + re75, family = tobit(Lower = 0, 
##     Upper = 10), data = lalonde, refresh = 0)
## 
## Coefficients: 
##                Estimate Std. Error z value Pr(>|z|)    
## (Intercept):1 4.647e+00  7.017e-02  66.223   <2e-16 ***
## (Intercept):2 1.589e+00  7.713e-03 206.072   <2e-16 ***
## treat         8.280e-01  3.795e-01   2.182   0.0291 *  
## re75          3.624e-04  4.489e-06  80.726   <2e-16 ***
## ---
## Signif. codes:  0 '***' 0.001 '**' 0.01 '*' 0.05 '.' 0.1 ' ' 1
## 
## Names of linear predictors: mu, loglink(sd)
## 
## Log-likelihood: -34510.66 on 37330 degrees of freedom
## 
## Number of Fisher scoring iterations: 21 
## 
## No Hauck-Donner effect found in any of the estimates
\end{verbatim}

As we don't consider the upper and lower bound, when all the predictors
are 0, log(re78+1) would be 4.647. If we condier the bounds, when all
the predictors are 0, log(re78+1) would be 1.589. Keep other indicator
the same, as one unit increase of treat, the log(re78+1) will increase
by 0.828. Keep otehr indicator the same, as one unit increase in re75,
the log(78+1) will increase by 3.624e-04.

\hypertarget{robust-linear-regression-using-the-t-model}{%
\subsection{15.8 Robust linear regression using the t
model}\label{robust-linear-regression-using-the-t-model}}

The folder \texttt{Congress} has the votes for the Democratic and
Republican candidates in each U.S. congressional district in 1988, along
with the parties' vote proportions in 1986 and an indicator for whether
the incumbent was running for reelection in 1988. For your analysis,
just use the elections that were contested by both parties in both
years.

\begin{Shaded}
\begin{Highlighting}[]
\NormalTok{congress }\OtherTok{\textless{}{-}} \FunctionTok{read.csv}\NormalTok{(}\StringTok{"congress.csv"}\NormalTok{, }\AttributeTok{header =}\NormalTok{ T)}
\NormalTok{c1988 }\OtherTok{\textless{}{-}} \FunctionTok{data.frame}\NormalTok{(}
    \AttributeTok{vote=}\NormalTok{congress}\SpecialCharTok{$}\NormalTok{v88\_adj,}
    \AttributeTok{pastvote=}\NormalTok{congress}\SpecialCharTok{$}\NormalTok{v86\_adj,}
    \AttributeTok{inc=}\NormalTok{congress}\SpecialCharTok{$}\NormalTok{inc88)}
\end{Highlighting}
\end{Shaded}

\hypertarget{a-2}{%
\subsubsection{(a)}\label{a-2}}

Fit a linear regression using \texttt{stan\_glm} with the usual
normal-distribution model for the errors predicting 1988 Democratic vote
share from the other variables and assess model fit.

\begin{Shaded}
\begin{Highlighting}[]
\NormalTok{m8 }\OtherTok{\textless{}{-}} \FunctionTok{stan\_glm}\NormalTok{(vote}\SpecialCharTok{\textasciitilde{}}\NormalTok{pastvote }\SpecialCharTok{+}\NormalTok{ inc, }\AttributeTok{data =}\NormalTok{ c1988, }\AttributeTok{refresh =} \DecValTok{0}\NormalTok{) }
\FunctionTok{summary}\NormalTok{(m8)}
\end{Highlighting}
\end{Shaded}

\begin{verbatim}
## 
## Model Info:
##  function:     stan_glm
##  family:       gaussian [identity]
##  formula:      vote ~ pastvote + inc
##  algorithm:    sampling
##  sample:       4000 (posterior sample size)
##  priors:       see help('prior_summary')
##  observations: 435
##  predictors:   3
## 
## Estimates:
##               mean   sd   10%   50%   90%
## (Intercept) 0.2    0.0  0.2   0.2   0.3  
## pastvote    0.5    0.0  0.5   0.5   0.6  
## inc         0.1    0.0  0.1   0.1   0.1  
## sigma       0.1    0.0  0.1   0.1   0.1  
## 
## Fit Diagnostics:
##            mean   sd   10%   50%   90%
## mean_PPD 0.5    0.0  0.5   0.5   0.5  
## 
## The mean_ppd is the sample average posterior predictive distribution of the outcome variable (for details see help('summary.stanreg')).
## 
## MCMC diagnostics
##               mcse Rhat n_eff
## (Intercept)   0.0  1.0  1671 
## pastvote      0.0  1.0  1648 
## inc           0.0  1.0  1652 
## sigma         0.0  1.0  2115 
## mean_PPD      0.0  1.0  3762 
## log-posterior 0.0  1.0  1776 
## 
## For each parameter, mcse is Monte Carlo standard error, n_eff is a crude measure of effective sample size, and Rhat is the potential scale reduction factor on split chains (at convergence Rhat=1).
\end{verbatim}

\begin{Shaded}
\begin{Highlighting}[]
\FunctionTok{pp\_check}\NormalTok{(m8)}
\end{Highlighting}
\end{Shaded}

\includegraphics{MA678-HW5_files/figure-latex/unnamed-chunk-10-1.pdf}

\hypertarget{b-2}{%
\subsubsection{(b)}\label{b-2}}

Fit the same sort of model using the \texttt{brms} package with a \(t\)
distribution, using the \texttt{brm} function with the student family.
Again assess model fit.

\begin{Shaded}
\begin{Highlighting}[]
\NormalTok{mb }\OtherTok{\textless{}{-}} \FunctionTok{brm}\NormalTok{(vote}\SpecialCharTok{\textasciitilde{}}\NormalTok{pastvote }\SpecialCharTok{+}\NormalTok{ inc, }\AttributeTok{data =}\NormalTok{ c1988, }\AttributeTok{family =} \StringTok{"student"}\NormalTok{, }\AttributeTok{chains =} \DecValTok{2}\NormalTok{, }\AttributeTok{iter =} \DecValTok{2000}\NormalTok{, }\AttributeTok{refresh =} \DecValTok{0}\NormalTok{)}
\end{Highlighting}
\end{Shaded}

\begin{verbatim}
## Compiling Stan program...
\end{verbatim}

\begin{verbatim}
## Start sampling
\end{verbatim}

\begin{Shaded}
\begin{Highlighting}[]
\FunctionTok{summary}\NormalTok{(mb)}
\end{Highlighting}
\end{Shaded}

\begin{verbatim}
##  Family: student 
##   Links: mu = identity; sigma = identity; nu = identity 
## Formula: vote ~ pastvote + inc 
##    Data: c1988 (Number of observations: 435) 
##   Draws: 2 chains, each with iter = 2000; warmup = 1000; thin = 1;
##          total post-warmup draws = 2000
## 
## Population-Level Effects: 
##           Estimate Est.Error l-95% CI u-95% CI Rhat Bulk_ESS Tail_ESS
## Intercept     0.22      0.02     0.19     0.26 1.00     1030     1145
## pastvote      0.55      0.03     0.48     0.61 1.00      996      924
## inc           0.09      0.01     0.08     0.11 1.00      921      940
## 
## Family Specific Parameters: 
##       Estimate Est.Error l-95% CI u-95% CI Rhat Bulk_ESS Tail_ESS
## sigma     0.05      0.00     0.05     0.06 1.00      865      970
## nu        6.22      3.05     3.39    12.54 1.00      888      918
## 
## Draws were sampled using sampling(NUTS). For each parameter, Bulk_ESS
## and Tail_ESS are effective sample size measures, and Rhat is the potential
## scale reduction factor on split chains (at convergence, Rhat = 1).
\end{verbatim}

\begin{Shaded}
\begin{Highlighting}[]
\FunctionTok{pp\_check}\NormalTok{(mb)}
\end{Highlighting}
\end{Shaded}

\begin{verbatim}
## Using 10 posterior draws for ppc type 'dens_overlay' by default.
\end{verbatim}

\includegraphics{MA678-HW5_files/figure-latex/unnamed-chunk-11-1.pdf}

\hypertarget{c-1}{%
\subsubsection{(c)}\label{c-1}}

Which model do you prefer?

I would prefer the student-t model since it fits the real data better
than the normal distribution model.

\hypertarget{robust-regression-for-binary-data-using-the-robit-model}{%
\subsection{15.9 Robust regression for binary data using the robit
model}\label{robust-regression-for-binary-data-using-the-robit-model}}

Use the same data as the previous example with the goal instead of
predicting for each district whether it was won by the Democratic or
Republican candidate.

\hypertarget{a-3}{%
\subsubsection{(a)}\label{a-3}}

Fit a standard logistic or probit regression and assess model fit.

\begin{Shaded}
\begin{Highlighting}[]
\NormalTok{c1988 }\OtherTok{\textless{}{-}}\NormalTok{ c1988}\SpecialCharTok{|\textgreater{}}
    \FunctionTok{mutate}\NormalTok{(}\AttributeTok{p =} \FunctionTok{ifelse}\NormalTok{(vote }\SpecialCharTok{\textgreater{}} \FloatTok{0.5}\NormalTok{, }\DecValTok{1}\NormalTok{, }\DecValTok{0}\NormalTok{))}
\NormalTok{mlog }\OtherTok{\textless{}{-}} \FunctionTok{stan\_glm}\NormalTok{(p}\SpecialCharTok{\textasciitilde{}}\NormalTok{pastvote }\SpecialCharTok{+}\NormalTok{ inc, }\AttributeTok{data =}\NormalTok{ c1988, }\AttributeTok{family =} \FunctionTok{binomial}\NormalTok{(}\AttributeTok{link =} \StringTok{"logit"}\NormalTok{), }\AttributeTok{refresh =} \DecValTok{0}\NormalTok{)}
\FunctionTok{summary}\NormalTok{(mlog)}
\end{Highlighting}
\end{Shaded}

\begin{verbatim}
## 
## Model Info:
##  function:     stan_glm
##  family:       binomial [logit]
##  formula:      p ~ pastvote + inc
##  algorithm:    sampling
##  sample:       4000 (posterior sample size)
##  priors:       see help('prior_summary')
##  observations: 435
##  predictors:   3
## 
## Estimates:
##               mean   sd   10%   50%   90%
## (Intercept) -5.8    1.3 -7.5  -5.8  -4.1 
## pastvote    11.7    2.6  8.4  11.6  15.2 
## inc          2.7    0.5  2.1   2.7   3.3 
## 
## Fit Diagnostics:
##            mean   sd   10%   50%   90%
## mean_PPD 0.6    0.0  0.6   0.6   0.6  
## 
## The mean_ppd is the sample average posterior predictive distribution of the outcome variable (for details see help('summary.stanreg')).
## 
## MCMC diagnostics
##               mcse Rhat n_eff
## (Intercept)   0.0  1.0  2157 
## pastvote      0.1  1.0  1984 
## inc           0.0  1.0  2487 
## mean_PPD      0.0  1.0  3513 
## log-posterior 0.0  1.0  1781 
## 
## For each parameter, mcse is Monte Carlo standard error, n_eff is a crude measure of effective sample size, and Rhat is the potential scale reduction factor on split chains (at convergence Rhat=1).
\end{verbatim}

\begin{Shaded}
\begin{Highlighting}[]
\FunctionTok{pp\_check}\NormalTok{(mlog)}
\end{Highlighting}
\end{Shaded}

\includegraphics{MA678-HW5_files/figure-latex/unnamed-chunk-12-1.pdf}

The logit model fits well, and it's better than the previous two models.

\hypertarget{b-3}{%
\subsubsection{(b)}\label{b-3}}

Fit a robit regression and assess model fit.

\begin{Shaded}
\begin{Highlighting}[]
\NormalTok{mr }\OtherTok{\textless{}{-}} \FunctionTok{brm}\NormalTok{(p}\SpecialCharTok{\textasciitilde{}}\NormalTok{pastvote}\SpecialCharTok{+}\NormalTok{inc, }\AttributeTok{data=}\NormalTok{c1988, }\AttributeTok{family=}\FunctionTok{student}\NormalTok{(}\AttributeTok{link=}\StringTok{"logit"}\NormalTok{))}
\end{Highlighting}
\end{Shaded}

\begin{verbatim}
## Compiling Stan program...
\end{verbatim}

\begin{verbatim}
## Start sampling
\end{verbatim}

\begin{verbatim}
## 
## SAMPLING FOR MODEL 'anon_model' NOW (CHAIN 1).
## Chain 1: 
## Chain 1: Gradient evaluation took 0.000139 seconds
## Chain 1: 1000 transitions using 10 leapfrog steps per transition would take 1.39 seconds.
## Chain 1: Adjust your expectations accordingly!
## Chain 1: 
## Chain 1: 
## Chain 1: Iteration:    1 / 2000 [  0%]  (Warmup)
## Chain 1: Iteration:  200 / 2000 [ 10%]  (Warmup)
## Chain 1: Iteration:  400 / 2000 [ 20%]  (Warmup)
## Chain 1: Iteration:  600 / 2000 [ 30%]  (Warmup)
## Chain 1: Iteration:  800 / 2000 [ 40%]  (Warmup)
## Chain 1: Iteration: 1000 / 2000 [ 50%]  (Warmup)
## Chain 1: Iteration: 1001 / 2000 [ 50%]  (Sampling)
## Chain 1: Iteration: 1200 / 2000 [ 60%]  (Sampling)
## Chain 1: Iteration: 1400 / 2000 [ 70%]  (Sampling)
## Chain 1: Iteration: 1600 / 2000 [ 80%]  (Sampling)
## Chain 1: Iteration: 1800 / 2000 [ 90%]  (Sampling)
## Chain 1: Iteration: 2000 / 2000 [100%]  (Sampling)
## Chain 1: 
## Chain 1:  Elapsed Time: 1.137 seconds (Warm-up)
## Chain 1:                0.853 seconds (Sampling)
## Chain 1:                1.99 seconds (Total)
## Chain 1: 
## 
## SAMPLING FOR MODEL 'anon_model' NOW (CHAIN 2).
## Chain 2: 
## Chain 2: Gradient evaluation took 6.1e-05 seconds
## Chain 2: 1000 transitions using 10 leapfrog steps per transition would take 0.61 seconds.
## Chain 2: Adjust your expectations accordingly!
## Chain 2: 
## Chain 2: 
## Chain 2: Iteration:    1 / 2000 [  0%]  (Warmup)
## Chain 2: Iteration:  200 / 2000 [ 10%]  (Warmup)
## Chain 2: Iteration:  400 / 2000 [ 20%]  (Warmup)
## Chain 2: Iteration:  600 / 2000 [ 30%]  (Warmup)
## Chain 2: Iteration:  800 / 2000 [ 40%]  (Warmup)
## Chain 2: Iteration: 1000 / 2000 [ 50%]  (Warmup)
## Chain 2: Iteration: 1001 / 2000 [ 50%]  (Sampling)
## Chain 2: Iteration: 1200 / 2000 [ 60%]  (Sampling)
## Chain 2: Iteration: 1400 / 2000 [ 70%]  (Sampling)
## Chain 2: Iteration: 1600 / 2000 [ 80%]  (Sampling)
## Chain 2: Iteration: 1800 / 2000 [ 90%]  (Sampling)
## Chain 2: Iteration: 2000 / 2000 [100%]  (Sampling)
## Chain 2: 
## Chain 2:  Elapsed Time: 0.803 seconds (Warm-up)
## Chain 2:                0.488 seconds (Sampling)
## Chain 2:                1.291 seconds (Total)
## Chain 2: 
## 
## SAMPLING FOR MODEL 'anon_model' NOW (CHAIN 3).
## Chain 3: 
## Chain 3: Gradient evaluation took 8.3e-05 seconds
## Chain 3: 1000 transitions using 10 leapfrog steps per transition would take 0.83 seconds.
## Chain 3: Adjust your expectations accordingly!
## Chain 3: 
## Chain 3: 
## Chain 3: Iteration:    1 / 2000 [  0%]  (Warmup)
## Chain 3: Iteration:  200 / 2000 [ 10%]  (Warmup)
## Chain 3: Iteration:  400 / 2000 [ 20%]  (Warmup)
## Chain 3: Iteration:  600 / 2000 [ 30%]  (Warmup)
## Chain 3: Iteration:  800 / 2000 [ 40%]  (Warmup)
## Chain 3: Iteration: 1000 / 2000 [ 50%]  (Warmup)
## Chain 3: Iteration: 1001 / 2000 [ 50%]  (Sampling)
## Chain 3: Iteration: 1200 / 2000 [ 60%]  (Sampling)
## Chain 3: Iteration: 1400 / 2000 [ 70%]  (Sampling)
## Chain 3: Iteration: 1600 / 2000 [ 80%]  (Sampling)
## Chain 3: Iteration: 1800 / 2000 [ 90%]  (Sampling)
## Chain 3: Iteration: 2000 / 2000 [100%]  (Sampling)
## Chain 3: 
## Chain 3:  Elapsed Time: 1.054 seconds (Warm-up)
## Chain 3:                0.749 seconds (Sampling)
## Chain 3:                1.803 seconds (Total)
## Chain 3: 
## 
## SAMPLING FOR MODEL 'anon_model' NOW (CHAIN 4).
## Chain 4: 
## Chain 4: Gradient evaluation took 6.2e-05 seconds
## Chain 4: 1000 transitions using 10 leapfrog steps per transition would take 0.62 seconds.
## Chain 4: Adjust your expectations accordingly!
## Chain 4: 
## Chain 4: 
## Chain 4: Iteration:    1 / 2000 [  0%]  (Warmup)
## Chain 4: Iteration:  200 / 2000 [ 10%]  (Warmup)
## Chain 4: Iteration:  400 / 2000 [ 20%]  (Warmup)
## Chain 4: Iteration:  600 / 2000 [ 30%]  (Warmup)
## Chain 4: Iteration:  800 / 2000 [ 40%]  (Warmup)
## Chain 4: Iteration: 1000 / 2000 [ 50%]  (Warmup)
## Chain 4: Iteration: 1001 / 2000 [ 50%]  (Sampling)
## Chain 4: Iteration: 1200 / 2000 [ 60%]  (Sampling)
## Chain 4: Iteration: 1400 / 2000 [ 70%]  (Sampling)
## Chain 4: Iteration: 1600 / 2000 [ 80%]  (Sampling)
## Chain 4: Iteration: 1800 / 2000 [ 90%]  (Sampling)
## Chain 4: Iteration: 2000 / 2000 [100%]  (Sampling)
## Chain 4: 
## Chain 4:  Elapsed Time: 1.033 seconds (Warm-up)
## Chain 4:                1.027 seconds (Sampling)
## Chain 4:                2.06 seconds (Total)
## Chain 4:
\end{verbatim}

\begin{verbatim}
## Warning: There were 3650 divergent transitions after warmup. See
## https://mc-stan.org/misc/warnings.html#divergent-transitions-after-warmup
## to find out why this is a problem and how to eliminate them.
\end{verbatim}

\begin{verbatim}
## Warning: There were 3 chains where the estimated Bayesian Fraction of Missing Information was low. See
## https://mc-stan.org/misc/warnings.html#bfmi-low
\end{verbatim}

\begin{verbatim}
## Warning: Examine the pairs() plot to diagnose sampling problems
\end{verbatim}

\begin{verbatim}
## Warning: The largest R-hat is 3.87, indicating chains have not mixed.
## Running the chains for more iterations may help. See
## https://mc-stan.org/misc/warnings.html#r-hat
\end{verbatim}

\begin{verbatim}
## Warning: Bulk Effective Samples Size (ESS) is too low, indicating posterior means and medians may be unreliable.
## Running the chains for more iterations may help. See
## https://mc-stan.org/misc/warnings.html#bulk-ess
\end{verbatim}

\begin{verbatim}
## Warning: Tail Effective Samples Size (ESS) is too low, indicating posterior variances and tail quantiles may be unreliable.
## Running the chains for more iterations may help. See
## https://mc-stan.org/misc/warnings.html#tail-ess
\end{verbatim}

\begin{Shaded}
\begin{Highlighting}[]
\FunctionTok{summary}\NormalTok{(mr)}
\end{Highlighting}
\end{Shaded}

\begin{verbatim}
## Warning: Parts of the model have not converged (some Rhats are > 1.05). Be
## careful when analysing the results! We recommend running more iterations and/or
## setting stronger priors.
\end{verbatim}

\begin{verbatim}
## Warning: There were 3650 divergent transitions after warmup. Increasing
## adapt_delta above 0.8 may help. See
## http://mc-stan.org/misc/warnings.html#divergent-transitions-after-warmup
\end{verbatim}

\begin{verbatim}
##  Family: student 
##   Links: mu = logit; sigma = identity; nu = identity 
## Formula: p ~ pastvote + inc 
##    Data: c1988 (Number of observations: 435) 
##   Draws: 4 chains, each with iter = 2000; warmup = 1000; thin = 1;
##          total post-warmup draws = 4000
## 
## Population-Level Effects: 
##           Estimate Est.Error l-95% CI u-95% CI Rhat Bulk_ESS Tail_ESS
## Intercept  -330.49    105.35  -533.94  -168.63 2.28        5       20
## pastvote     22.61     10.99    10.37    38.91 3.88        4       11
## inc        1036.80    506.64   287.05  2073.75 2.94        5       16
## 
## Family Specific Parameters: 
##       Estimate Est.Error l-95% CI u-95% CI Rhat Bulk_ESS Tail_ESS
## sigma     0.00      0.00     0.00     0.00 1.02      277       NA
## nu        1.00      0.00     1.00     1.00 3.44        4       12
## 
## Draws were sampled using sampling(NUTS). For each parameter, Bulk_ESS
## and Tail_ESS are effective sample size measures, and Rhat is the potential
## scale reduction factor on split chains (at convergence, Rhat = 1).
\end{verbatim}

\begin{Shaded}
\begin{Highlighting}[]
\FunctionTok{pp\_check}\NormalTok{(mr)}
\end{Highlighting}
\end{Shaded}

\begin{verbatim}
## Using 10 posterior draws for ppc type 'dens_overlay' by default.
\end{verbatim}

\includegraphics{MA678-HW5_files/figure-latex/unnamed-chunk-13-1.pdf}

\hypertarget{c-2}{%
\subsubsection{(c)}\label{c-2}}

Which model do you prefer?

By observing the pp graph, I prefer the logit model.

\hypertarget{model-checking-for-count-data}{%
\subsection{15.14 Model checking for count
data}\label{model-checking-for-count-data}}

The folder \texttt{RiskyBehavior} contains data from a study of behavior
of couples at risk for HIV; see Exercise 15.1.

\hypertarget{a-4}{%
\subsubsection{(a)}\label{a-4}}

Fit a Poisson regression predicting number of unprotected sex acts from
baseline HIV status. Perform predictive simulation to generate 1000
datasets and record the percentage of observations that are equal to 0
and the percentage that are greater than 10 (the third quartile in the
observed data) for each. Compare these to the observed value in the
original data.

\begin{Shaded}
\begin{Highlighting}[]
\NormalTok{mp }\OtherTok{\textless{}{-}} \FunctionTok{stan\_glm}\NormalTok{(fupacts}\SpecialCharTok{\textasciitilde{}}\NormalTok{bs\_hiv, }\AttributeTok{data =}\NormalTok{ risky, }\AttributeTok{family =} \FunctionTok{poisson}\NormalTok{(}\AttributeTok{link =} \StringTok{"log"}\NormalTok{), }\AttributeTok{refresh =} \DecValTok{0}\NormalTok{)}
\FunctionTok{summary}\NormalTok{(mp)}
\end{Highlighting}
\end{Shaded}

\begin{verbatim}
## 
## Model Info:
##  function:     stan_glm
##  family:       poisson [log]
##  formula:      fupacts ~ bs_hiv
##  algorithm:    sampling
##  sample:       4000 (posterior sample size)
##  priors:       see help('prior_summary')
##  observations: 434
##  predictors:   2
## 
## Estimates:
##                  mean   sd   10%   50%   90%
## (Intercept)     2.9    0.0  2.9   2.9   2.9 
## bs_hivpositive -0.6    0.0 -0.7  -0.6  -0.6 
## 
## Fit Diagnostics:
##            mean   sd   10%   50%   90%
## mean_PPD 16.5    0.3 16.1  16.5  16.8 
## 
## The mean_ppd is the sample average posterior predictive distribution of the outcome variable (for details see help('summary.stanreg')).
## 
## MCMC diagnostics
##                mcse Rhat n_eff
## (Intercept)    0.0  1.0  2769 
## bs_hivpositive 0.0  1.0  2256 
## mean_PPD       0.0  1.0  2871 
## log-posterior  0.0  1.0  1490 
## 
## For each parameter, mcse is Monte Carlo standard error, n_eff is a crude measure of effective sample size, and Rhat is the potential scale reduction factor on split chains (at convergence Rhat=1).
\end{verbatim}

\begin{Shaded}
\begin{Highlighting}[]
\FunctionTok{pp\_check}\NormalTok{(mp)}
\end{Highlighting}
\end{Shaded}

\includegraphics{MA678-HW5_files/figure-latex/unnamed-chunk-14-1.pdf}

\begin{Shaded}
\begin{Highlighting}[]
\NormalTok{pp }\OtherTok{\textless{}{-}} \FunctionTok{posterior\_predict}\NormalTok{(mp, }\AttributeTok{draw =} \DecValTok{1000}\NormalTok{)}
\NormalTok{obs\_p }\OtherTok{\textless{}{-}} \FunctionTok{data.frame}\NormalTok{(}
    \AttributeTok{num0 =} \FunctionTok{apply}\NormalTok{(pp, }\DecValTok{2}\NormalTok{, }\ControlFlowTok{function}\NormalTok{(x) }\FunctionTok{mean}\NormalTok{(x}\SpecialCharTok{==}\DecValTok{0}\NormalTok{)),}
    \AttributeTok{num10 =} \FunctionTok{apply}\NormalTok{(pp, }\DecValTok{2}\NormalTok{, }\ControlFlowTok{function}\NormalTok{(x) }\FunctionTok{mean}\NormalTok{(x}\SpecialCharTok{\textgreater{}=}\DecValTok{10}\NormalTok{)))}

\FunctionTok{ggplot}\NormalTok{(}\AttributeTok{data =}\NormalTok{ obs\_p, }\FunctionTok{aes}\NormalTok{(}\AttributeTok{x =}\NormalTok{ num0))}\SpecialCharTok{+}
    \FunctionTok{geom\_histogram}\NormalTok{(}\FunctionTok{aes}\NormalTok{(}\AttributeTok{x =}\NormalTok{ num0))}\SpecialCharTok{+}
    \FunctionTok{geom\_vline}\NormalTok{(}\FunctionTok{aes}\NormalTok{(}\AttributeTok{xintercept=}\FunctionTok{mean}\NormalTok{(risky}\SpecialCharTok{$}\NormalTok{fupacts }\SpecialCharTok{==} \DecValTok{0}\NormalTok{)), }\AttributeTok{linetype =} \StringTok{"dashed"}\NormalTok{)}\SpecialCharTok{+}
    \FunctionTok{labs}\NormalTok{(}\AttributeTok{x =} \StringTok{"number of 0s"}\NormalTok{)}
\end{Highlighting}
\end{Shaded}

\begin{verbatim}
## `stat_bin()` using `bins = 30`. Pick better value with `binwidth`.
\end{verbatim}

\includegraphics{MA678-HW5_files/figure-latex/unnamed-chunk-15-1.pdf}

\begin{Shaded}
\begin{Highlighting}[]
\FunctionTok{ggplot}\NormalTok{(}\AttributeTok{data =}\NormalTok{ obs\_p, }\FunctionTok{aes}\NormalTok{(}\AttributeTok{x =}\NormalTok{ num10))}\SpecialCharTok{+}
    \FunctionTok{geom\_histogram}\NormalTok{(}\FunctionTok{aes}\NormalTok{(}\AttributeTok{x =}\NormalTok{ num10))}\SpecialCharTok{+}
    \FunctionTok{geom\_vline}\NormalTok{(}\FunctionTok{aes}\NormalTok{(}\AttributeTok{xintercept=}\FunctionTok{mean}\NormalTok{(risky}\SpecialCharTok{$}\NormalTok{fupacts }\SpecialCharTok{\textgreater{}=} \DecValTok{10}\NormalTok{)), }\AttributeTok{linetype =} \StringTok{"dashed"}\NormalTok{)}\SpecialCharTok{+}
    \FunctionTok{labs}\NormalTok{(}\AttributeTok{x =} \StringTok{"number of 10s"}\NormalTok{)}\SpecialCharTok{+}
    \FunctionTok{xlim}\NormalTok{(}\FunctionTok{c}\NormalTok{(}\DecValTok{0}\NormalTok{,}\DecValTok{1}\NormalTok{))}
\end{Highlighting}
\end{Shaded}

\begin{verbatim}
## `stat_bin()` using `bins = 30`. Pick better value with `binwidth`.
\end{verbatim}

\begin{verbatim}
## Warning: Removed 2 rows containing missing values (`geom_bar()`).
\end{verbatim}

\includegraphics{MA678-HW5_files/figure-latex/unnamed-chunk-15-2.pdf}

\hypertarget{b-4}{%
\subsubsection{(b)}\label{b-4}}

Repeat (a) using a negative binomial (overdispersed Poisson) regression.

\begin{Shaded}
\begin{Highlighting}[]
\NormalTok{mnb }\OtherTok{\textless{}{-}} \FunctionTok{stan\_glm}\NormalTok{(fupacts}\SpecialCharTok{\textasciitilde{}}\NormalTok{bs\_hiv, }\AttributeTok{data =}\NormalTok{ risky, }\AttributeTok{family =} \FunctionTok{neg\_binomial\_2}\NormalTok{(}\AttributeTok{link =} \StringTok{"log"}\NormalTok{), }\AttributeTok{refresh =} \DecValTok{0}\NormalTok{)}
\FunctionTok{summary}\NormalTok{(mnb)}
\end{Highlighting}
\end{Shaded}

\begin{verbatim}
## 
## Model Info:
##  function:     stan_glm
##  family:       neg_binomial_2 [log]
##  formula:      fupacts ~ bs_hiv
##  algorithm:    sampling
##  sample:       4000 (posterior sample size)
##  priors:       see help('prior_summary')
##  observations: 434
##  predictors:   2
## 
## Estimates:
##                         mean   sd   10%   50%   90%
## (Intercept)            2.9    0.1  2.8   2.9   3.0 
## bs_hivpositive        -0.6    0.2 -0.9  -0.6  -0.3 
## reciprocal_dispersion  0.3    0.0  0.3   0.3   0.4 
## 
## Fit Diagnostics:
##            mean   sd   10%   50%   90%
## mean_PPD 16.6    2.0 14.1  16.5  19.1 
## 
## The mean_ppd is the sample average posterior predictive distribution of the outcome variable (for details see help('summary.stanreg')).
## 
## MCMC diagnostics
##                       mcse Rhat n_eff
## (Intercept)           0.0  1.0  3589 
## bs_hivpositive        0.0  1.0  3910 
## reciprocal_dispersion 0.0  1.0  3941 
## mean_PPD              0.0  1.0  3820 
## log-posterior         0.0  1.0  1897 
## 
## For each parameter, mcse is Monte Carlo standard error, n_eff is a crude measure of effective sample size, and Rhat is the potential scale reduction factor on split chains (at convergence Rhat=1).
\end{verbatim}

\begin{Shaded}
\begin{Highlighting}[]
\FunctionTok{pp\_check}\NormalTok{(mnb)}
\end{Highlighting}
\end{Shaded}

\includegraphics{MA678-HW5_files/figure-latex/unnamed-chunk-16-1.pdf}

\hypertarget{c-3}{%
\subsubsection{(c)}\label{c-3}}

Repeat (b), also including ethnicity and baseline number of unprotected
sex acts as inputs.

\begin{Shaded}
\begin{Highlighting}[]
\NormalTok{ppc }\OtherTok{\textless{}{-}} \FunctionTok{posterior\_predict}\NormalTok{(mnb, }\AttributeTok{draw =} \DecValTok{1000}\NormalTok{)}
\NormalTok{obs\_nb }\OtherTok{\textless{}{-}} \FunctionTok{data.frame}\NormalTok{(}
    \AttributeTok{num0 =} \FunctionTok{apply}\NormalTok{(ppc, }\DecValTok{2}\NormalTok{, }\ControlFlowTok{function}\NormalTok{(x) }\FunctionTok{mean}\NormalTok{(x}\SpecialCharTok{==}\DecValTok{0}\NormalTok{)),}
    \AttributeTok{num10 =} \FunctionTok{apply}\NormalTok{(ppc, }\DecValTok{2}\NormalTok{, }\ControlFlowTok{function}\NormalTok{(x) }\FunctionTok{mean}\NormalTok{(x}\SpecialCharTok{\textgreater{}=}\DecValTok{10}\NormalTok{)))}

\FunctionTok{ggplot}\NormalTok{(}\AttributeTok{data =}\NormalTok{ obs\_nb, }\FunctionTok{aes}\NormalTok{(}\AttributeTok{x =}\NormalTok{ num0))}\SpecialCharTok{+}
    \FunctionTok{geom\_histogram}\NormalTok{(}\FunctionTok{aes}\NormalTok{(}\AttributeTok{x =}\NormalTok{ num0))}\SpecialCharTok{+}
    \FunctionTok{geom\_vline}\NormalTok{(}\FunctionTok{aes}\NormalTok{(}\AttributeTok{xintercept=}\FunctionTok{mean}\NormalTok{(risky}\SpecialCharTok{$}\NormalTok{fupacts }\SpecialCharTok{==} \DecValTok{0}\NormalTok{)), }\AttributeTok{linetype =} \StringTok{"dashed"}\NormalTok{)}\SpecialCharTok{+}
    \FunctionTok{labs}\NormalTok{(}\AttributeTok{x =} \StringTok{"number of 0s"}\NormalTok{)}\SpecialCharTok{+}
    \FunctionTok{xlim}\NormalTok{(}\FunctionTok{c}\NormalTok{(}\DecValTok{0}\NormalTok{,}\DecValTok{1}\NormalTok{))}
\end{Highlighting}
\end{Shaded}

\begin{verbatim}
## `stat_bin()` using `bins = 30`. Pick better value with `binwidth`.
\end{verbatim}

\begin{verbatim}
## Warning: Removed 2 rows containing missing values (`geom_bar()`).
\end{verbatim}

\includegraphics{MA678-HW5_files/figure-latex/unnamed-chunk-17-1.pdf}

\begin{Shaded}
\begin{Highlighting}[]
\FunctionTok{ggplot}\NormalTok{(}\AttributeTok{data =}\NormalTok{ obs\_nb, }\FunctionTok{aes}\NormalTok{(}\AttributeTok{x =}\NormalTok{ num10))}\SpecialCharTok{+}
    \FunctionTok{geom\_histogram}\NormalTok{(}\FunctionTok{aes}\NormalTok{(}\AttributeTok{x =}\NormalTok{ num10))}\SpecialCharTok{+}
    \FunctionTok{geom\_vline}\NormalTok{(}\FunctionTok{aes}\NormalTok{(}\AttributeTok{xintercept=}\FunctionTok{mean}\NormalTok{(risky}\SpecialCharTok{$}\NormalTok{fupacts }\SpecialCharTok{\textgreater{}=} \DecValTok{10}\NormalTok{)), }\AttributeTok{linetype =} \StringTok{"dashed"}\NormalTok{)}\SpecialCharTok{+}
    \FunctionTok{labs}\NormalTok{(}\AttributeTok{x =} \StringTok{"number of 10s"}\NormalTok{)}\SpecialCharTok{+}
    \FunctionTok{xlim}\NormalTok{(}\FunctionTok{c}\NormalTok{(}\DecValTok{0}\NormalTok{,}\DecValTok{1}\NormalTok{))}
\end{Highlighting}
\end{Shaded}

\begin{verbatim}
## `stat_bin()` using `bins = 30`. Pick better value with `binwidth`.
\end{verbatim}

\begin{verbatim}
## Warning: Removed 2 rows containing missing values (`geom_bar()`).
\end{verbatim}

\includegraphics{MA678-HW5_files/figure-latex/unnamed-chunk-17-2.pdf}

\hypertarget{summarizing-inferences-and-predictions-using-simulation}{%
\subsection{15.15 Summarizing inferences and predictions using
simulation}\label{summarizing-inferences-and-predictions-using-simulation}}

Exercise 15.7 used a Tobit model to fit a regression with an outcome
that had mixed discrete and continuous data. In this exercise you will
revisit these data and build a two-step model: (1) logistic regression
for zero earnings versus positive earnings, and (2) linear regression
for level of earnings given earnings are positive. Compare predictions
that result from each of these models with each other.

\begin{Shaded}
\begin{Highlighting}[]
\NormalTok{lalonde }\OtherTok{\textless{}{-}} \FunctionTok{read.dta}\NormalTok{(}\StringTok{"NSW\_dw\_obs.dta"}\NormalTok{)}
\NormalTok{lalonde}\SpecialCharTok{$}\NormalTok{bin78 }\OtherTok{\textless{}{-}} \FunctionTok{ifelse}\NormalTok{(lalonde}\SpecialCharTok{$}\NormalTok{re78 }\SpecialCharTok{\textgreater{}} \DecValTok{0}\NormalTok{, }\DecValTok{1}\NormalTok{, }\DecValTok{0}\NormalTok{)}
\NormalTok{m1 }\OtherTok{=} \FunctionTok{stan\_glm}\NormalTok{(bin78 }\SpecialCharTok{\textasciitilde{}}\NormalTok{ treat }\SpecialCharTok{+}\NormalTok{ re75, }\AttributeTok{data =}\NormalTok{ lalonde, }\AttributeTok{family =} \FunctionTok{binomial}\NormalTok{(}\AttributeTok{link=}\StringTok{"logit"}\NormalTok{), }\AttributeTok{refresh =} \DecValTok{0}\NormalTok{)}
\FunctionTok{pp\_check}\NormalTok{(m1)}
\end{Highlighting}
\end{Shaded}

\includegraphics{MA678-HW5_files/figure-latex/unnamed-chunk-18-1.pdf}

\begin{Shaded}
\begin{Highlighting}[]
\NormalTok{m2 }\OtherTok{=} \FunctionTok{stan\_glm}\NormalTok{(}\FunctionTok{log}\NormalTok{(re78) }\SpecialCharTok{\textasciitilde{}}\NormalTok{ treat }\SpecialCharTok{+}\NormalTok{ re75, }\AttributeTok{data=}\NormalTok{lalonde[lalonde}\SpecialCharTok{$}\NormalTok{bin78}\SpecialCharTok{==}\DecValTok{1}\NormalTok{,], }\AttributeTok{refresh=}\DecValTok{0}\NormalTok{)}
\FunctionTok{pp\_check}\NormalTok{(m2)}
\end{Highlighting}
\end{Shaded}

\includegraphics{MA678-HW5_files/figure-latex/unnamed-chunk-19-1.pdf}

a

\end{document}
